\documentclass[10pt]{article}

\usepackage{hyperref} 
\usepackage{graphicx} 
\usepackage{fancybox} 
\usepackage{rotating} 
\usepackage{color} 
%%\usepackage{DIN_A4} 
%%\usepackage{here}

\ifx\MARaBOU\undefined
\def\MARaBOU{MAR\lower.5ex\hbox{a}BO\kern-.5em\lower.5ex\hbox{O}\kern-.1em U}%
\fi

\newcommand{\samefootnote}{$^\alph{footnote}$}
\newcommand{\samempfootnote}{$^\alph{mpfootnote}$}

\definecolor{lightblue}{rgb}{.68,.88,.90}

\newcommand{\blue}[1]{\colorbox{lightblue}{\texttt{#1}}}
\newcommand{\yellow}[1]{\colorbox{yellow}{\texttt{#1}}}
\newcommand{\redt}[1]{\textcolor{red}{\texttt{#1}}}
\newcommand{\greent}[1]{\textcolor{green}{\texttt{#1}}}
\newcommand{\bluet}[1]{\textcolor{blue}{\texttt{#1}}}

\newenvironment{boxed}
	{\begin{Sbox}\begin{minipage}[t]}
	{\end{minipage}\end{Sbox}\fbox{\TheSbox}}
	
\newenvironment{yellowboxed}
	{\begin{Sbox}\begin{minipage}[t]}
	{\end{minipage}\end{Sbox}\colorbox{yellow}{\TheSbox}}
	
\newenvironment{blueboxed}
	{\begin{Sbox}\begin{minipage}[t]}
	{\end{minipage}\end{Sbox}\colorbox{lightblue}{\TheSbox}}
	

\setlength{\parindent}{0pt}

\begin{document}
\begin{titlepage}
\title{How to perform event building in a MINIBALL experiment}
\author{R. Lutter}
\maketitle
\vfill
\begin{abstract}
\setlength{\parindent}{0pt}
\begin{center}
This document describes how to perform event building and to generate \texttt{ROOT} trees from med data in a \mbox{Miniball} experiment.
\end{center}
\end{abstract}
\vfill
\end{titlepage}
\newpage
\tableofcontents
\newpage
\section{Online data acquisition}\label{OnlineDataAcquisition}
Any data in a MINIBALL experiment will be produced by connecting to the \texttt{MBS} subsystem and reading raw data from the harware modules
in \texttt{VME} and \texttt{CAMAC} crates \cite{DaqInstructions}.
Online data are then written to disk using the \texttt{MED} format (\texttt{MBS} event data) \cite{MedStructure}.\\
Fig. \ref{DataAcquisition} shows data flow as well as software components used to acquire experimental data.\\

Following files are needed for this step:
\begin{center}
\begin{itemize}
\setlength{\rightmargin}{1em}%
\setlength{\leftmargin}{2em}%
\setlength{\itemsep}{0pt}%
\setlength{\parskip}{1mm}%
\setlength{\partopsep}{0pt}%
\setlength{\parsep}{0pt}%
\setlength{\topsep}{0pt}%
\item	\yellow{DgfReadout.c}\\
	user's readout function given by hardware definitions in \yellow{Config.C},\\
	auto-generated during config step
\item	\yellow{DgfAnalyze.cxx}\\
	methods and functions to setup the \texttt{ROOT} system and to decode raw data from \texttt{MBS},\\
	auto-generated during the config step
\item	\yellow{udef/ProcessEvent.cxx}\\
	consistency checks and accumulation of diagnostic histograms,\\
	has to be provided by user
\end{itemize}
\end{center}
\begin{figure}[H]
\centerline{\includegraphics[width=.75\linewidth]{DataAcquisition}}
\caption{Online data acquisition}
\label{DataAcquisition}
\end{figure}
\newpage
\section{Filling hitBuffers from raw data}\label{FillingHB}
\subsection{Extracting hits from DGF-4C data}\label{ExtractDgfHits}
\subsection{Extracting hits from CAEN ADC/TDC data}\label{ExtractCaenHits}
\newpage
\section{Event building}\label{EventBuilding}
In a second step one may create meaningful events from raw data read from \texttt{MED} files.
These events may then be written to \texttt{ROOT} trees and used in replay sessions afterwards.
Fig. \ref{EventBuilder} shows how to do the event building.

\begin{figure}[H]
\centerline{\includegraphics[width=.75\linewidth]{EventBuilder}}
\caption{Event building}
\label{EventBuilder}
\end{figure}
Steps to be performed for the event building process:
\begin{center}
\begin{itemize}
\setlength{\rightmargin}{1em}%
\setlength{\leftmargin}{2em}%
\setlength{\itemsep}{0pt}%
\setlength{\parskip}{1mm}%
\setlength{\partopsep}{0pt}%
\setlength{\parsep}{0pt}%
\setlength{\topsep}{0pt}%
\item	in \yellow{SetCppIfdefs.C}: activate event building\\
	\hspace*{.2\linewidth}\yellow{dgf->MakeDefined("\_\_EVENT\_BUILDING\_ON\_\_", kTRUE, "Start event building");}
\item	in \yellow{Config.C}: define an event class to be used for event building\\
	\hspace*{.2\linewidth}\yellow{dgf->IncludeUserClass("udef", "MbEvent.cxx", kTRUE);}\\
	when doing the config step for the first time this will generate a template class \texttt{MbEvent};\\
	you then have to modify this class according to your needs
\item	provide an event building algorithm: \texttt{TUsrEvtReadout::BuildEvent()} in \yellow{udef/BuildEvent.cxx}
(table \ref{TabEvbAlgorithm}, fig. \ref{FigEvbAlgorithm})
\item	in \yellow{.rootrc} file: add an entry \blue{TMrbAnalyze.CoincWindow: N}
\item	perform config, compile and link the whole stuff, run \texttt{C\_analyze}
\end{itemize}
\end{center}
\newpage
\subsection{Inserting time stamps in ADC/TDC data}\label{InsertingTimeStamps}
\begin{center}
\begin{table}[H]
{\scriptsize
\begin{yellowboxed}{\linewidth}
\verb+  Bool_t TUsrEvtReadout::InsertTimeStamp(Int_t TsIdx, Int_t TsChannel, Int_t StartIdx, Int_t NofHbx) {+\\
\verb-  //________________________________________________________________[C++ METHOD]-\\
\verb+  //////////////////////////////////////////////////////////////////////////////+\\
\verb+  // Name:         TUsrEvtReadout::InsertTimeStamp+\\
\verb+  // Purpose:      Insert time stamps into adc data+\\
\verb+  // Arguments:    Int_t TsIdx       -- index of timestamping buffer+\\
\verb+  //               Int_t TsChannel   -- number of timestamping channel+\\
\verb+  //               Int_t StartIdx    -- index of adc buffer to start with+\\
\verb+  //               Int_t NofHbx      -- number of buffers to process+\\
\verb+  // Results:      kTRUE/kFALSE+\\
\verb+  // Description:  Inserts time stamps from timestamping dgf into adc buffers.+\\
\verb+  //               Time stamps in dgf buffer are sorted by time,+\\
\verb+  //               adc/tdc buffers contain single channel data grouped by event number.+\\
\verb+  //               This algorithm inserts time stamps one by one in adc channel data;+\\
\verb+  //               channels belonging to one event (i.e. having the same event number)+\\
\verb+  //               will be marked with same time stamp.+\\
\verb+  // Keywords:     +\\
\verb+  //////////////////////////////////////////////////////////////////////////////+\\
\verb+  +\\
\verb+    TUsrHBX *tsHbx = this->GetHBX(TsIdx);                   // connect to hitbuffer of time stamping dgf+\\
\verb+  +\\
\verb-    for (Int_t i = StartIdx; i < StartIdx + NofHbx; i++) {  // loop over adcs/tdcs-\\
\verb+      tsHbx->ResetIndex();                                  // reset to first item in ts-dgf buffer+\\
\verb+      TUsrHit * tsHit = tsHbx->FindHit(TsChannel);          // fetch first item (= time stamp)+\\
\verb+      if(!tsHit) break;                                     // no time stamps in this buffer (should never be)+\\
\verb+      TUsrHBX * cHbx = this->GetHBX(i);                     // connect to current adc/tdc buffer+\\
\verb+      cHbx->ResetIndex();                                   // reset to first item in buffer+\\
\verb+      TUsrHit * cHit = cHbx->NextHit();                     // fetch first adc/tdc hit (= channel data)+\\
\verb+      while(tsHit && cHit) {                                // step thru buffers+\\
\verb+        Int_t evtNo = cHit->GetEventNumber();               // save current event number from adc/tdc hit+\\
\verb+        while(cHit->GetEventNumber() == evtNo) {            // as long as event number doesn't change:+\\
\verb+          cHit->SetChannelTime(tsHit);                      //    insert current time stamp in adc hit+\\
\verb+          cHit = cHbx->NextHit();+\\
\verb+          if (!cHit) break;                                 // end of adc/tdc buffer reached+\\
\verb+        }+\\
\verb+        tsHit = tsHbx->FindHit(TsChannel);                  // event number in adc/tdc buffer has changed:+\\
\verb+      }                                                     //    get next time stamp from ts-dgf buffer+\\
\verb+    }+\\
\verb+    return(kTRUE);+\\
\verb+  }+
\end{yellowboxed}}
\caption{Algorithm used for time stamp insertion}
\label{TabTsiAlgorithm}
\end{table}
\end{center}
\subsection{Event building algorithm}\label{EventBuildingAlgorithm}
\begin{center}
\begin{table}[H]
{\scriptsize
\begin{yellowboxed}{\linewidth}
\verb+  Bool_t TUsrEvtReadout::BuildEvent() {+\\
\verb-  //________________________________________________________________[C++ METHOD]-\\
\verb+  //////////////////////////////////////////////////////////////////////////////+\\
\verb+  // Name:        TUsrEvtReadout::BuildEvent+\\
\verb+  // Purpose:     Event building+\\
\verb+  // Arguments:   --+\\
\verb+  // Results:     kTRUE/kFALSE+\\
\verb+  // Exceptions:  +\\
\verb+  // Description: Loops over all hit buffers,+\\
\verb+  //              collects hits within window 'coincWindow'+\\
\verb+  //              stores hits in events of type MbEvent+\\
\verb+  //              then calls method MbEvent::WriteToTree() for each event+\\
\verb+  //              Strategy:+\\
\verb+  //                a binary tree TBtree is used to compare events+\\
\verb+  //                method TBtree::FindObject() calls MbEvent::Compare()+\\
\verb+  //                MbEvent::IsEqual() then tests if time stamps are 'equal' within 'coincWindow'+\\
\verb+  //                hits belonging to an event are stored in a hit buffer of type TClonesArray,+\\
\verb+  //                to get the benefits of TClonesArray these hit buffers have to be static+\\
\verb+  //                they will be allocated only once and reused afterwards. +\\
\verb+  // Keywords:     +\\
\verb+  //////////////////////////////////////////////////////////////////////////////+\\
\verb+  +\\
\verb+    TBtree evtsS;+\\
\verb+    TBtree * evts = &evtsS;+\\
\verb+  +\\
\verb+  // reset hit buffers+\\
\verb+    TObjArrayIter epI(&poolOfMbEvents);+\\
\verb+    while(MbEvent * evt = (MbEvent *) epI.Next()) evt->Reset();+\\
\verb+  +\\
\verb+  // fill binary tree, create events if necessary+\\
\verb+    Int_t evtCount = 0;+\\
\verb-    for (Int_t i = kIdxCluster1; i < kIdxCluster1 + kNofSevts; i++) {    // loop over all hit buffers-\\
\verb+      if (i == kIdxTSCluster) continue;           // discard time stamper+\\
\verb+      TUsrHBX *hbx = this->GetHBX(i);             // pointer to current hit buffer+\\
\verb+      if (hbx) {+\\
\verb+        Int_t hitNo = 0;+\\
\verb+        hbx->ResetIndex();                        // reset buffer index to first hit+\\
\verb+        while (TUsrHit *hit = hbx->NextHit()) {   // loop over hits of this buffer+\\
\verb+          if (hit->GetChannelTime() == 0) continue;              // time stamp = 0 -> junk data+\\
\verb+          tmpEvent.SetEventTime(hit->GetChannelTime());          // insert time stamp into temp event+\\
\verb+          MbEvent * evt = (MbEvent *) evts->FindObject(&tmpEvent);   // compare it with events already there+\\
\verb+          if (evt == NULL) {                                     // no match+\\
\verb+            if (evtCount < poolOfMbEvents.GetEntriesFast()) {    // try to get a event from pool+\\
\verb+              evt = (MbEvent *) poolOfMbEvents[evtCount];+\\
\verb+            } else {+\\
\verb+              evt = new MbEvent();                // create a new one+\\
\verb+              poolOfMbEvents.Add(evt);            // add it to pool+\\
\verb+            }+\\
\verb+            evt->Reset();                         // reset event time, hit buffer etc+\\
\verb+            evt->SetEventTime(hit->GetChannelTime());  // insert time stamp+\\
\verb+            evt->SetCoincWindow(coincWdw);        // should know about time window+\\
\verb+            evts->Add(evt);                       // add new event to binary tree+\\
\verb-            evtCount++;-\\
\verb+          }+\\
\verb+          evt->AddHit(hit);                       // append current hit to event+\\
\verb-          hitNo++;-\\
\verb+        }+\\
\verb+      }+\\
\verb+    }+\\
\verb+  +\\
\verb+  // call method MbEvent::WriteToTree() for all events+\\
\verb+	  TBtreeIter evI(evts);+\\
\verb+	  while(MbEvent * evt = (MbEvent *) evI.Next()) evt->WriteToTree();  // write event data to tree+\\
\verb+	  evtsS.Clear();                                // clear entries in binary tree+\\
\verb+	  return(kTRUE);+\\
\verb+  }+
\end{yellowboxed}}
\caption{Algorithm used for the event building process}
\label{TabEvbAlgorithm}
\end{table}
\end{center}
\newpage
\begin{figure}[H]
\centerline{\includegraphics[width=\linewidth]{EvbAlgorithm}}
\caption{Event builder: schematic diagram}
\label{FigEvbAlgorithm}
\end{figure}
\newpage
\section{Replay session}\label{ReplaySession}
Once events have been built and tree data have been written to \texttt{ROOT} file one may start a replay session
by reading these pre-sorted data. Fig. \ref{ReplayTreeData} shows the data flow for such a replay session.
The gain in speed is about a factor of 10 as compared to replaying raw data from \texttt{MED} files.

\begin{figure}[H]
\centerline{\includegraphics[width=.75\linewidth]{ReplaySession}}
\caption{Offline replay session}
\label{ReplayTreeData}
\end{figure}

User has to provide a method \yellow{MbEvent::Analyze()} containing his analysis. A wrapper class \yellow{MbExp} may be used
to establish pointer arrays pointing to different event components such as cores, segments, channels, etc (table \ref{MbExp}).

\begin{center}
\begin{table}[H]
{\scriptsize
\begin{yellowboxed}{\linewidth}
\verb-  //______________________________________________________[C++ CLASS DEFINITION]-\\
\verb+  //////////////////////////////////////////////////////////////////////////////+\\
\verb+  // Name:           MbExp+\\
\verb+  // Purpose:        Wrapper class to access MbEvent data+\\
\verb+  // Description:    Organizes MbEvent data in a more "experiment-oriented" way.+\\
\verb+  //                 Data may be accessed via cluster, core, and segment numbers (dgf),+\\
\verb+  //                 via adc and channel numbers, resp.+\\
\verb+  //                 Performs boundary check if requested.+\\
\verb+  //                 Indices start with 0.+\\
\verb+  // Keywords:+\\
\verb+  //////////////////////////////////////////////////////////////////////////////+\\
\verb+  +\\
\verb+  class MbExp {+\\
\verb+    +\\
\verb+    public:+\\
\verb+      MbExp(MbEvent * Event = NULL) { Init(Event); };    // default ctor+\\
\verb+      virtual ~MbExp() {};                // default dtor+\\
\verb+  +\\
\verb+      TUsrHit * Core(Int_t Cluster, Int_t Core, Bool_t BoundaryCheck = kFALSE);+\\
\verb+      TUsrHit * Segment(Int_t Cluster, Int_t Core, Int_t Seg, Bool_t BoundaryCheck = kFALSE);+\\
\verb+      TUsrHit * Particle(Int_t Dgf, Int_t Channel, Bool_t BoundaryCheck = kFALSE);+\\
\verb+      TUsrHit * Time(Int_t Tdc, Int_t Channel, Bool_t BoundaryCheck = kFALSE);+\\
\verb+  +\\
\verb+      void Init(MbEvent * Event = NULL); // initialize ptr arrays with data from MbEvent+\\
\verb+  +\\
\verb+    protected:+\\
\verb+      TUsrHit * fCores[kNofDgfClusters][kNofDgfCoresPerCluster];+\\
\verb+      TUsrHit * fSegs[kNofDgfClusters][kNofDgfCoresPerCluster][kNofDgfSegsPerCore];+\\
\verb+      TUsrHit * fPart[kNofAdcs][kNofChannelsPerAdc];+\\
\verb+      TUsrHit * fTime[kNofTdcs][kNofChannelsPerTdc];+\\
\verb+  +\\
\verb+    protected:+\\
\verb+      TUsrHit fEmptyHit;+\\
\verb+  };+\\
\end{yellowboxed}}
\caption{Wrapper class to access event components}
\label{MbExp}
\end{table}
\end{center}
\newpage
\begin{thebibliography}{9}
\bibitem{DaqInstructions} See also: "Instructions how to use the DAQ in a MINIBALL experiment",\\
\verb+http://www.bl.physik.uni-muenchen.de/marabou/html/doc/DaqInstructions.pdf+
\bibitem{MedStructure} For details see: "MED Data Structure",\\
\verb+http://www.bl.physik.uni-muenchen.de/marabou/html/doc/MedStructure.pdf+
\end{thebibliography}
\end{document}
