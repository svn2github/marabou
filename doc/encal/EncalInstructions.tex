\documentclass[10pt]{article}

\usepackage{hyperref} 
\usepackage{graphicx} 
\usepackage{fancybox} 
\usepackage{rotating} 
\usepackage{color} 
\usepackage{DIN_A4} 
\usepackage{here}

\ifx\MARaBOU\undefined
\def\MARaBOU{MAR\lower.5ex\hbox{a}BO\kern-.5em\lower.5ex\hbox{O}\kern-.1em U}%
\fi

\newcommand{\samefootnote}{$^\alph{footnote}$}
\newcommand{\samempfootnote}{$^\alph{mpfootnote}$}

\definecolor{lightblue}{rgb}{.68,.88,.90}
\definecolor{lightgray}{rgb}{.85,.85,.85}

\newcommand{\blue}[1]{\colorbox{lightblue}{\texttt{#1}}}
\newcommand{\gray}[1]{\colorbox{lightgray}{\texttt{#1}}}
\newcommand{\yellow}[1]{\colorbox{yellow}{\texttt{#1}}}
\newcommand{\redt}[1]{\textcolor{red}{#1}}
\newcommand{\redtt}[1]{\textcolor{red}{\texttt{#1}}}
\newcommand{\greent}[1]{\textcolor{green}{#1}}
\newcommand{\greentt}[1]{\textcolor{green}{\texttt{#1}}}
\newcommand{\bluet}[1]{\textcolor{blue}{#1}}
\newcommand{\bluett}[1]{\textcolor{blue}{\texttt{#1}}}
\newcommand{\boldt}[1]{\textbf{\texttt{#1}}}

\newenvironment{boxed}
	{\begin{Sbox}\begin{minipage}[t]}
	{\end{minipage}\end{Sbox}\fbox{\TheSbox}}
	
\newenvironment{yellowboxed}
	{\begin{Sbox}\begin{minipage}[t]}
	{\end{minipage}\end{Sbox}\colorbox{yellow}{\TheSbox}}
	
\newenvironment{blueboxed}
	{\begin{Sbox}\begin{minipage}[t]}
	{\end{minipage}\end{Sbox}\colorbox{lightblue}{\TheSbox}}

\newenvironment{grayboxed}
	{\begin{Sbox}\begin{minipage}[t]}
	{\end{minipage}\end{Sbox}\colorbox{lightgray}{\TheSbox}}
	

\setlength{\parindent}{0pt}

\begin{document}
\begin{titlepage}
\title{Instructions how to perform an energy calibration in the \MARaBOU{} environment}
\author{R. Lutter}
\maketitle
\vfill
\begin{abstract}
\setlength{\parindent}{0pt}
\begin{center}
This document describes how to perform an energy calibration by use of \MARaBOU{}'s \texttt{MacroBrowser} and the \texttt{ROOT} macro \blue{Encal.C}.
\end{center}
\end{abstract}
\vfill
\end{titlepage}
\newpage
\section{How to perform an energy calibration}\vspace{3mm}

To perform an energy calibration (gamma or particle) call the \texttt{MacroBrowser}:\\

\hspace*{.2\linewidth}\yellow{MacroBrowser}\\

A menu will then pop up showing several \texttt{ROOT} macros. Choose \blue{Encal.C} from this list.\\

You'll get a form (fig. \ref{EncalForm}) to specify parameters needed to start the calibration:\\

\begin{center}
\begin{itemize}
\setlength{\rightmargin}{1em}%
\setlength{\leftmargin}{2em}%
\setlength{\itemsep}{0pt}%
\setlength{\parskip}{1mm}%
\setlength{\partopsep}{0pt}%
\setlength{\parsep}{0pt}%
\setlength{\topsep}{0pt}%
\item	\blue{Histogram file (.root)}\\
	Click on the folder button and choose the \texttt{ROOT} file containing your calibration spectra.
	Then select some histograms from the list below either by \textbf{range} (first \dots last) or one by one (\textbf{single}).
	Press \gray{apply} to accept selection, \gray{clear} to reset list.
\item	\blue{Calibration source}\\
	Choose \gray{Co60} or \gray{Eu152} for gammas, \gray{TripleAlpha} for particles
\item	\blue{Calibration energies}\\
	Enter file containing calibration energies. Default is \texttt{\$MARABOU/data/encal/energies.dat}.
	See \ref{EngFileFormat} for a format description.
\item	\blue{Calibration data}\\
	Where to write calibration data (extension \texttt{.cal}).\\
	Calibration data will be formatted according to \texttt{ROOT}'s resource format (ROOT object \texttt{TEnv}, \ref{CalFileFormat});
\item	\blue{Calibration results}\\
	Where to write detailed calibration results (extension \texttt{.res}).
	Same formatting rules as for \texttt{.cal} files apply (\ref{ResFileFormat}).
\item	\blue{Fit results}\\
	\texttt{ROOT} file where to store histograms together with fit data as well as liniear regression data (extension \texttt{.root}).
\item	\blue{Clear output files}\\
	Click \gray{yes} to remove existing files (\texttt{.cal} and \texttt{.res}) on start.
	If \gray{no} is chosen new data will be appended to existing files, existing items will be overwritten with new values.
\item	\blue{Precalibration file}\\
	To get an \gray{Eu152} calibration you have to preform a \gray{Co60} calibration step first in order to assign gamma energies
	to peak centers.
	The name of the \gray{Co60} calibration file from a previous run has to be given here.
\item	\blue{Lower/Upper limit}\\
	Limits of the display/fit region (channels)
\item	\blue{Threshold for peak finder}\\
	Relative height of peaks taken into account given by percentage of maximum peak
	\hfill(continued on next page)
\newpage
\item	\blue{Sigma}\\
	\texttt{sigma} value for gaussian fit
\item	\blue{Fit mode}\\
	Either \gray{gauss} or \gray{gauss+tail}: how to fit lines in the histogram.
\item	\blue{Fit grouping}\\
	You may fit \gray{single} peaks  or an \gray{ensemble} of peaks (may give better results for \gray{Co60}
	and \gray{TripleAlpha} calibrations).
\item	\blue{Background}\\
	You have the choice between a \gray{const} or \gray{linear} background.
\item	\blue{Range for peak fit}\\
	Set lower and upper limits for the peak fit region (single peak  or group fitting), given in units of \texttt{sigma}.
\item	\blue{Display results}\\
	Check button \gray{step} to stop after each fit, check \gray{2dim} to display a two-dim histogram containing re-calibrated
	histograms line by line.
\item	\blue{Verbose}\\
	Choosing button \gray{yes} will output some diagnostics to \texttt{stdout}.
\end{itemize}
\newpage
\end{center}
\begin{figure}[H]
\centerline{\includegraphics[width=.9\linewidth]{EncalForm}}
\caption{\texttt{Encal.C}: how to perform an energy calibration}
\label{EncalForm}
\end{figure}
\newpage
Press \blue{Execute} to start the calibration.\vspace{.5cm}

A new window will pop up showing fit results (fig. \ref{FitDisplay}).\vspace{.5cm}

The upper part shows peak centroids (\bluet{blue markers}) as
returned by the peak finder together with the gaussian fit (\redt{red curve}).

The lower part shows the linear regression results
when assigning alpha lines 5.157 MeV ($^{239}$Pu), 5.486 MeV ($^{241}$Am), and 5.865 MeV ($^{244}$Cm) to peaks, respectively.

The title bar below the graphs shows the resulting calibration formula \gray{E(x) = a$_0$ + a$_1$ * x}.\vspace{1cm}
\begin{figure}[H]
\centerline{\includegraphics[width=\linewidth]{FitDisplay}}
\caption{\texttt{Encal.C}: fit and calibration results}
\label{FitDisplay}
\end{figure}

If you have chosen \gray{step} mode you now have to press one of the buttons at the bottom edge of the display:
\begin{center}
\begin{itemize}
\setlength{\rightmargin}{1em}%
\setlength{\leftmargin}{2em}%
\setlength{\itemsep}{0pt}%
\setlength{\parskip}{1mm}%
\setlength{\partopsep}{0pt}%
\setlength{\parsep}{0pt}%
\setlength{\topsep}{0pt}%
\item	\gray{accept}
	add calibration data to output file (\texttt{.cal}) and continue with next histogram.
\item	\gray{discard}
	discard calibration and continue with next histogram.
\item	\gray{stop}
	leave calibration loop and return to \texttt{MacroBrowser}'s menu (fig. \ref{EncalForm}).
\item	\gray{quit}
	close files and exit from \texttt{ROOT}.
\end{itemize}\vspace{.5cm}
\end{center}
\newpage
If you checked the \gray{2dim} button you will get a two-dimensional histogram showing calibrated
histograms line by line (fig. \ref{Display2dim}).

From this plot you can see immediately that calibration of histogram \texttt{hRawAs1}
didn't give satisfactory results. Now try the following:
\begin{center}
\begin{itemize}
\setlength{\rightmargin}{1em}%
\setlength{\leftmargin}{2em}%
\setlength{\itemsep}{0pt}%
\setlength{\parskip}{1mm}%
\setlength{\partopsep}{0pt}%
\setlength{\parsep}{0pt}%
\setlength{\topsep}{0pt}%
\item	Go back to \texttt{MacroBrowser}'s menu (fig. \ref{EncalForm})
\item	Reset the histogram list by pressing \gray{clear}
\item	Select histogram \texttt{hRawAs1} from the list (button \gray{apply})
\item	Change flag \blue{Clear output files} to \gray{no}
\item	Change the \blue{sigma} value to \gray{2}
\item	Press \blue{Execute} to start over
\end{itemize}\vspace{.5cm}
\end{center}
The new fit data for histogram \texttt{hRawAs1} will then overwrite the existing ones without changing other entries. You then end up with
fig. \ref{Display2dimCorr}.
\newpage
\begin{figure}[H]
\centerline{\includegraphics[width=.9\linewidth]{Display2dim}}
\caption{\texttt{Encal.C}: 2-dim plot of calibrated histograms, first try}
\label{Display2dim}
\end{figure}
\begin{figure}[H]
\centerline{\includegraphics[width=.9\linewidth]{Display2dimCorr}}
\caption{\texttt{Encal.C}: 2-dim plot after re-fitting}
\label{Display2dimCorr}
\end{figure}
\newpage
\section{File formats}
\subsection{Calibration data (extension \texttt{.cal})}\label{CalFileFormat}
A calibration data file (extension \texttt{.cal}) generated by \texttt{Encal.C}
is formatted according to \texttt{ROOT}'s resource format (see \texttt{ROOT} documentation, object \texttt{TEnv}).
It consists of a header followed by entries for each histogram.

\begin{center}
\begin{tabular}{|l|l|}
\hline
Header	&	\texttt{Calib.ROOTFile: <HistoFile>.root}		\\
	&	\texttt{Calib.Source: Co60 or Eu152 or TripleAlpha}		\\
	&	\texttt{Calib.Energies: <path to file containing calib energies>}		\\
	&	\texttt{Calib.NofHistograms: <NH>}		\\
\hline
Entry	&	\texttt{Calib.<HistoName>.Xmin:	<lower limit>}	\\
(1 per histo) &	\texttt{Calib.<HistoName>.Xmax:	<upper limit>}	\\
	&	\texttt{Calib.<HistoName>.Gain: <gain a$_1$>}	\\
	&	\texttt{Calib.<HistoName>.Offset: <offset a$_0$>}	\\
\hline
\end{tabular}
\end{center}

This file may be read in directly to the data acquisition program by use of method \texttt{TMrbAnalyze::ReadCalibrationFromFile()}.
\subsection{Detailed calibration results (extension \texttt{.res})}\label{ResFileFormat}
In addition to the calibration data file (\ref{CalFileFormat}) detailed fit results will be written toa file with extension \texttt{.res}.
Its format is similar to that of the \texttt{.cal} file but contains much more information.

\begin{center}
\begin{tabular}{|l|l|}
\hline
Header	&	\texttt{Calib.ROOTFile: <HistoFile>.root}		\\
	&	\texttt{Calib.Source: Co60 or Eu152 or TripleAlpha}		\\
	&	\texttt{Calib.Energies: <path to file containing calib energies>}		\\
	&	\texttt{Calib.NofHistograms: <NH>}		\\
	&	\texttt{Calib.Emin: <minimum energy over all histos>}		\\
	&	\texttt{Calib.Emax: <maximum energy>}		\\
	&	\texttt{Calib.Xmin: <minimum channel>}		\\
	&	\texttt{Calib.Xmax: <maximum channel>}		\\
\hline
Histo entry	&	\texttt{Calib.<HistoName>.Xmin:	<lower limit>}	\\
(1 per histo) &	\texttt{Calib.<HistoName>.Xmax:	<upper limit>}	\\
	&	\texttt{Calib.<HistoName>.NofPeaks: <NP>}	\\
	&	\texttt{Calib.<HistoName>.Gain: <gain a$_1$>}	\\
	&	\texttt{Calib.<HistoName>.Offset: <offset a$_0$>}	\\
	&	\texttt{Calib.<HistoName>.FitOk: TRUE, FALSE, or AUTO}	\\
\hline
Peak entry	&	\texttt{Calib.<HistoName>.Peak.<I>.X:	<centroid peak finder>}	\\
(1 per peak per histo) &	\texttt{Calib.<HistoName>.Peak.<I>.Xfit:	<centroid gaussian fit>}	\\
	&	\texttt{Calib.<HistoName>.Peak.<I>.Xerr:	<centroid error>}	\\
	&	\texttt{Calib.<HistoName>.Peak.<I>.Y:	<amplitude peak finder>}	\\
	&	\texttt{Calib.<HistoName>.Peak.<I>.Yfit:	<amplitude gaussian fit>}	\\
	&	\texttt{Calib.<HistoName>.Peak.<I>.Yerr:	<amplitude error>}	\\
	&	\texttt{Calib.<HistoName>.Peak.<I>.Chi2:	<chisquare gaussian fit>}	\\
\hline
\end{tabular}
\end{center}
\subsection{Calibration energies}\label{EngFileFormat}
Calibration energies have to be provided as \texttt{ROOT} resources (see \texttt{ROOT} documentation, object \texttt{TEnv}).

\begin{center}
\begin{tabular}{|l|l|}
\hline
Header	&	\texttt{Calib.NofCalibrations: <NC>}		\\
	&	\texttt{Calib.SourceNames: <Source1>:<Source2>:\dots:<SourceNC>}		\\
\hline
Entry	&	\texttt{Calib.<SourceN>.NofLines: <NL>}		\\
(1 per line per source) &	\texttt{Calib.<SourceN>.Line.<I>.E: <energy>}		\\
	&	\texttt{Calib.<SourceN>.Line.<I>.Eerr: <error>}		\\
\hline
\end{tabular}
\end{center}

\end{document}
