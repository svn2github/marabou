\documentclass[10pt]{article}

\setlength{\textwidth}{15cm}
\setlength{\textheight}{24cm}
\setlength{\oddsidemargin}{0cm}
\setlength{\evensidemargin}{16cm}
\addtolength{\evensidemargin}{-\oddsidemargin}
\addtolength{\evensidemargin}{-\textwidth}
\setlength{\topmargin}{-2cm}

\usepackage{hyperref} 
\usepackage{graphicx} 
\usepackage{fancybox} 
\usepackage{rotating} 
\usepackage{color} 

\ifx\MARaBOU\undefined
\def\MARaBOU{MAR\lower.5ex\hbox{a}BO\kern-.5em\lower.5ex\hbox{O}\kern-.1em U}%
\fi

\newcommand{\samefootnote}{$^\alph{footnote}$}
\newcommand{\samempfootnote}{$^\alph{mpfootnote}$}

\definecolor{lightblue}{rgb}{.68,.88,.90}

\newcommand{\blue}[1]{\colorbox{lightblue}{\texttt{#1}}}
\newcommand{\yellow}[1]{\colorbox{yellow}{\texttt{#1}}}
\newcommand{\red}[1]{\colorbox{red}{\texttt{#1}}}
\newcommand{\green}[1]{\colorbox{green}{\texttt{#1}}}
\newcommand{\redt}[1]{\textcolor{red}{\texttt{#1}}}
\newcommand{\greent}[1]{\textcolor{green}{\texttt{#1}}}
\newcommand{\bluet}[1]{\textcolor{blue}{\texttt{#1}}}
\newcommand{\boldt}[1]{\textbf{\texttt{#1}}}

\newenvironment{boxed}
	{\begin{Sbox}\begin{minipage}[t]}
	{\end{minipage}\end{Sbox}\fbox{\TheSbox}}
	
\newenvironment{yellowboxed}
	{\begin{Sbox}\begin{minipage}[t]}
	{\end{minipage}\end{Sbox}\colorbox{yellow}{\TheSbox}}
	
\newenvironment{blueboxed}
	{\begin{Sbox}\begin{minipage}[t]}
	{\end{minipage}\end{Sbox}\colorbox{lightblue}{\TheSbox}}
	

\setlength{\parindent}{0pt}

\begin{document}
\begin{titlepage}
\title{How to set up the PPC}
\author{R. Lutter}
\maketitle
\vfill
\begin{abstract}
\setlength{\parindent}{0pt}
\begin{center}
This document describes cabling, configuring, and booting the PowerPC\\
in a \mbox{MINIBALL} experiment.
\end{center}
\end{abstract}
\vfill
\end{titlepage}
\newpage
\tableofcontents
\newpage
\section{Introduction}\label{Introduction}
Normally the PPC should be set up properly, the LynxOs operating system should be working fine, any cabling should be in order. In case of hang-ups simply pressing the \red{RESET} button should solve the problem. Thus \dots
\begin{center}
\redt{Don't change anything under normal conditions!\\
Don't touch a running system!}
 \end{center}

This manual is intended to give some insights into PPC internals. You may thus check settings in case system got completely stuck.

\section{Cabling}\label{Cabling}
\begin{minipage}[b]{.6\linewidth}
Adjoining figure shows the front panel of the PPC.
\begin{itemize}
\setlength{\rightmargin}{1em}%
\setlength{\leftmargin}{2em}%
\setlength{\itemsep}{0pt}%
\setlength{\parskip}{1mm}%
\setlength{\partopsep}{0pt}%
\setlength{\parsep}{0pt}%
\setlength{\topsep}{0pt}%
\item	\red{Reset} button
\item	\yellow{local ethernet}, 10 Mbits/s,\\
	on-board,\\
	system name \texttt{le}
\item	\blue{serial i/o}
\item	\green{fast ethernet}, 100 Mbits/s,\\
	piggy-back position 2\\
	system name \texttt{e2}
\end{itemize}

\subsection{Connecting to ethernet}
As there is now (June 2007) a fast ethernet interface in the PPC port \green{e2} (piggy-back module in position 2) should be connected to port \green{GB1}
\footnote{Be aware: \texttt{GB1} is assigned to \texttt{eth1} in the daq server internally, while \texttt{GB2} is assigned to \texttt{eth0}!}
at the rear side of the daq server.

The local ethernet \yellow{le} is still working, but using the fast ethernet will improve the thruput when
transfering data to the daq computer.
\subsection{Connecting the serial i/o}
The \blue{serial i/o} should always be connected to the serial port at the rear side of the daq server.
It will be used for maintenance only. In case of booting problems it is the only way to inspect and
configure the PPC's boot section. This can be managed even from remote.
\end{minipage}
\begin{minipage}[b]{.3\linewidth}
\includegraphics[height=.8\textheight]{RIO2.jpg}
\end{minipage}
\newpage
\section{Serial I/O}
To connect to the PPC via the serial line call program \texttt{minicom} in the daq server \texttt{pcepuis16}.
Pressing the \red{Reset} button should give an output like this:\\

\verb+PPC Boot Rev 3.60 created Thu Sep 16 13:39:36 1999+\\
\verb+ +\\
\verb+  Module Type: 8062WA  rev: C   serial: 1358+\\
\verb+  Host CPU: PPC604e   ver: 31.02   speed: 300 Mhz+\\
\verb+  PCI Bridge: IBM27-82660 (Lanai/Kauai) rev: 02+\\
\verb-  Memory Size: 32+0+0+0+0+0+0+0 = 32 Mbytes-\\
\verb+               EDO RAM  60 ns   ECC enabled+\\
\verb+  L2 CACHE:  1 Mbyte SSRAM+\\
\verb+  FPROM: Two banks of FUJI29F016 installed+\\
\verb+  Ethernet Address: 00:80:a2:00:8d:5e+\\
\verb+ +\\
\verb+Entering boot diagnostics+\\
\verb+ +\\
\verb+ 0    Check System Memory (0x00000000 - 0x01effffc)          0  SKIPPED+\\
\verb+ 1    Check Interrupt Handler                                0  OK+\\
\verb+ 2    Check PCI Bridge                                       1  OK+\\
\verb+ 3    Check Cache and MMU                                    1  OK+\\
\verb+ 4    Check MK48T08 RTC and NVRAM                            1  OK+\\
\verb+ 5    Check SIC6351 Interrupt Controller                     1  OK+\\
\verb+ 6    Check PC87312 Super IO                                 0  OK+\\
\verb+ 7    Check ZCIO8536 Micro Timers                            1  OK+\\
\verb+ 8    Check FIFO's (0 - 7)                                   1  OK+\\
\verb+ 9    Check DS1620 Digital Thermometer                       1  OK+\\
\verb+10    Check PCI devices                                      0  OK+\\
\verb+11    Check AM79C970 Ethernet Controller  (PCI slot 0)       1  OK+\\
\verb+12    Check PMC #1 Extension  (PCI slot 2)                   0  OK+\\
\verb+13    Check PMC #2 Extension  (PCI slot 1)                   0  OK+\\
\verb+14    Check VME Interface                                    0  OK+\\
\verb+ +\\
\verb+enter 'SPACE' to stop auto boot...+\\
\verb+enter 'b' or '@' to boot OS...+\\
\verb+2+\\
\verb+**************************************************+\\
\verb+*   PPC_Mon RTPC8062 monitor - version 3.6       *+\\
\verb+*     CES SA Copyright 1995,1996,1997,1998       *+\\
\verb+**************************************************+\\
\verb+ +\\
\verb+PPC_Mon>+\\

Pressing the space bar will stop the boot procedure, otherwise it will boot automatically after 3s.\\
Entering \blue{boot} will start the boot sequence.\\

You may leave \texttt{minicom} by typing \yellow{CTRL+A Shift+Z Q}.
\newpage
\section{Boot settings}
To inspect boot settings type \blue{show boot} at the \texttt{PPCMon} prompt:\\

\verb+BOOT parameters+\\\label{PPC-Boot}
\verb+ +\\
\verb+   boot_flags [afmnN] : a+\\
\verb+   boot_device [s<1-2>d<0-7><a-d>, le, e<0-2>, tty<0-1>, fp] : e2+\\
\verb+   boot_file :+\\
\verb+   boot_rootfs [s<1-2>d<0-7><a-d>, rd] : rd+\\
\verb+   boot_delay : 3 sec+\\
\verb+   boot_address : 4000+\\
\verb+   boot_size : 1e00000+\\
\verb+   boot_fast [ y n ] : n+\\
\verb+   boot_filetype [ binary xcoff elf lynx prep srec ] : binary+\\
\verb+   boot_mode : 50+\\

One may change these parameters by use of command \blue{set boot} - it will step thru the list
of parameters and ask for a value. \redt{Be careful! Enter exactly the above values!}\\

The only modification may be to switch to the local (slow) ethernet interface by replacing \texttt{boot\_device:\,e2} by \texttt{boot\_device:\,le}.\\

To inspect ethernet settings type \blue{show inet}:\\

\verb+INTERNET parameters+\\\label{PPC-Inet}
\verb++\\
\verb+   inet_host [dotted decimal] : 192.168.1.10+\\
\verb+   inet_server [dotted decimal] : 192.168.1.1+\\
\verb+   inet_broadcast [dotted decimal] : 0.0.0.0+\\
\verb+   inet_mask [dotted decimal] : 0.0.0.0+\\
\verb+   inet_protocol [ arpa bootp ] : bootp+\\
\verb+   inet_mount :+\\

These parameters will be sent by the daq server via \texttt{DHCP}. Don't modify. The only param
which has to be checked is \texttt{inet\_protocol:\,bootp}. If it shows \texttt{arpa} you have to change it
back to \texttt{bootp} via command \blue{set inet\_protocol}.

After having checked everything type \blue{boot} to boot the PPC.\\
Then leave \texttt{minicom} by typing \yellow{CTRL+A Shift+Z Q}.

\newpage
\section{DHCP settings}
The daq computer \texttt{pcepuis16} acts as \texttt{DHCP/BOOTP} server for the PPC.\\

\texttt{DHCP} settings are defined in file \verb+/etc/dhcpd.conf+ on the server:\\

\verb+#+\\
\verb+# DHCP Server Configuration file.+\\
\verb+# Miniball config+\\
\verb+# R. Lutter Apr-2007+\\
\verb+#+\\
\verb++\\
\verb+use-host-decl-names on;+\\
\verb+ddns-update-style none;+\\
\verb+deny client-updates;+\\
\verb++\\
\verb+authoritative;+\\
\verb++\\
\verb+subnet 192.168.1.0 netmask 255.255.255.0 {+\\
\verb+    option subnet-mask 255.255.255.0;+\\
\verb+    option routers 192.168.1.1;+\\
\verb+    option broadcast-address 192.168.1.255;+\\
\verb+    option domain-name-servers 192.168.1.1;+\\
\verb+    option ip-forwarding off;+\\
\verb+    next-server 192.168.1.1;+\\
\verb+    range 192.168.1.128 192.168.1.200;+\\
\verb+}+\\
\verb++\\
\verb+group rio2 {+\\
\verb+  next-server 192.168.1.1;+\\
\verb+  server-name "minidaq";+\\
\verb++\\
\verb+  host ppc-0-amd {+\\
\verb+	hardware ethernet 00:80:A2:00:8D:5E;+\\
\verb+	fixed-address ppc-0;+\\
\verb+  filename "lynx/boot.rio2-amd0";+\\
\verb+  }+\\
\verb+  host ppc-0-feth {+\\
\verb+	hardware ethernet 00:80:A2:00:18:A8;+\\
\verb+	fixed-address ppc-0;+\\
\verb+  filename "lynx/boot.rio2-feth0";+\\
\verb+  }+\\
\verb+}+\\

As one can see there are entries for both interfaces, the local one (\texttt{amd0}) as well as
the fast ethernet (\texttt{feth0}), both resulting in a PPC named \texttt{ppc-0}. So booting
will depend on the boot settings in the PPC only (see \ref{PPC-Boot}).
\newpage
\section{Mounting file systems}
Following file systems will be mounted at boot time by the PPC:\\

\begin{tabular}{ll}
\verb+/dev/rd0+ & root file system \\
\verb+minidaq:/usr/diskless/lynx/RIO2_2.5 mounted on /nfs+ & LynxOs\\
\verb+minidaq:/usr/diskless/lynx/RIO2_2.5/mbs mounted on /mbs+ & MBS\\
\verb+minidaq:/usr/daq mounted on /usr/daq+ & MARaBOU\\
\verb+minidaq:/home mounted on /home+ & user's home dirs\\
\verb+minidaq:/mbdata mounted on /mbdata+ & miniball data\\
\end{tabular}\\

To modify and/or add settings you really have to be an expert!\\
Changes will apply to files
\verb+/nfs/net_boot+ in the PPC and \verb+/etc/exports+ in the daq server, respectively.





\end{document}
